\documentclass{article}
\usepackage[spanish]{babel}
\usepackage[utf8]{inputenc}

\title{Test de Verosimilitud Generalizada}
\date{21 de Noviembre 2018}

\begin{document}

\maketitle


\section{Prueba}

Para una muestra $ X_1,...X_n $  de una m.a.s con densidad:\par
\begin{center}
$f(x,\theta), \theta \in \Omega  $
\end{center} \par

\begin{flushleft}
Nosotros podemos contrastar \par

$H_0: \theta \in \Omega_0$ \par
$H_1: \theta \in \Omega_1 = \Omega - \Omega_0$\par

\end{flushleft}

\subsection{Test de razón de verosimilitud} 

Sup L $ (x_1,...x_n, \theta)$ \par

$$  \lambda =  \frac{\theta \subset \Omega_0}{Sup_{\theta \in \Omega_1} L (x_1,...,x_n, \theta)} = \frac{Bajo H_0}{  A^*}$$
*Con los estimadores de maxima verosimilitud.

$ 0 \leq \lambda \leq 1 $
\\

\begin{flushleft}
Si $ \lambda $ se aproxima a 1 puede afirmarse que la hipotesis $H_0$ se halla avalado por la muestra, ocurriendo lo contrario si $ \lambda $ se aproxima a 0. \\
Para $n$ suficientemente grande, siendo $k$ el número de dimensiones de $\Omega$ y $r$ el de $\Omega_0$, si $H_0$ es cierta. 

\end{flushleft}

$$ -2\ln\lambda \sim X_{(k-r)}^2$$

\end{document}
